\documentclass[12pt]{article}
\usepackage{hyperref}
\usepackage{cite}

\begin{document}
\title{CSE441 Semester Project: FunCoin}
\author{Rob Kelly\\Sean Turner\\Randy Van Why}
\maketitle

\section{Background}
In October of 2008, a white paper\cite{nakamoto:bitcoin} describing Bitcoin, a decentralized peer-to-peer electronic cash system, was published through cryptography mailing lists.

\section{OpenSSL}
The group used OpenSSL to handle most of the cryptographic functionality. OpenSSL provides a nice interface for performing digital signitures using DSA.

\section{Approach}
The basic approach we took was to familiarize ourselves with the Bitcoin paper\cite{nakamoto:bitcoin} and the Bitcoin Developer Guide \cite{dev:guide}. 

\subsection{Block Chain}

\section{Discussion}\label{future}
\subsection{Future Work}\label{work}
Some work that can be done to our implementation includes extending it from a simple client/server model to a true peer-to-peer cryptocurrency, making it more modular, so that interested parties may substitute their own implementations of key components such as the block chain and the proof of work in order to gain better understanding of the underlying techniques, and implement true transactions. 

Additional future projects include: implementing a wallet, considering different mining techniques altogether, and following on modularity, possibly making it into a library that can be used to roll your own simplified fun coin.

\subsection{Unsolved Problems}\label{unsolved}
Our biggest unsolved problem (and opportunity for Section \ref{future}) is to decentralize for a more accurate portrayal of how real cryptocurrencies function. Additionally, 

\bibliographystyle{abbrv}
\bibliography{report}

\end{document}

\documentclass[12pt]{article}
\usepackage{hyperref}
\usepackage{cite}

\begin{document}
\title{CSE441 Semester Project: FunCoin}
\author{Rob Kelly\\Sean Turner\\Randy Van Why}
\maketitle

<<<<<<< HEAD
\section{Approach}
\subsection{Motivation}
The goal of this project was to provide a pedagogical overview of cryptocurrencies on the whole
as well as attempt to discuss a theoretical implementation of a simple cryptocurrency for
demonstration purposes. In order to discuss our simplified model, the group took a look at
the most popular cryptocurrency implementation: Bitcoin.

\subsection{The BitCoin Implementation: An Overview}
The most natural starting point was for the group to think about the coin itself.
In Bitcoin-like cryptocurrencies, the coins that can be spent and mined are not
like traditional coins or electronic currency.

While it is obvious that Bitcoins are not physical coins, a surprising fact about
the currency is that Bitcoins are not even single digital entities or files at all.
Instead, Bitcoins are tracked and stored as transaction histories in a larger entity
known as the blockchain. An individual 

=======
\section{Background}
In October of 2008, a white paper\cite{nakamoto:bitcoin} describing Bitcoin, a decentralized peer-to-peer electronic cash system, was published through cryptography mailing lists.

\section{OpenSSL}
The group used OpenSSL to handle most of the cryptographic functionality. OpenSSL provides a nice interface for performing digital signitures using DSA.

\bibliographystyle{abbrv}
\bibliography{report}
>>>>>>> c11783ffbf51196c46aaa182b60af7a8393ecbdd
\end{document}
